\documentclass{tufte-handout}

\usepackage{amsmath}
\usepackage{graphicx}
\setkeys{Gin}{width=\linewidth,totalheight=\textheight,keepaspectratio}
%%\usepackage{url} -- blows things up quite nicely
%%\usepackage{natbib} -- Included by style file
\usepackage{alltt}
\usepackage{semantic}

\usepackage{mcaption}

\title{HNRS 190 -- Honors Tutorial}
\author{Security, Privacy and Trustworthiness in the Digital Age}

\makeatletter
\ifthenelse{\NOT\boolean{@tufte@xetex}}
  {%
    \IfFileExists{bergamo.sty}{\usepackage[osf]{bergamo}}{}% Bembo clone
    \IfFileExists{chantill.sty}{\usepackage{chantill}}{}% Gill Sans clone
  }{% We're using XeTeX -- load fontspec, etc.
    \usepackage{fontspec}
    \usepackage{xltxtra}
    \setromanfont{Bergamo}% Bembo clone
    \setsansfont{Chantilly}% Gill Sans clone
    \setmonofont{Bitstream Vera Sans Mono}%
  }
\makeatother

% The following package makes prettier tables.  We're all about the bling!
\usepackage{booktabs}

% The units package provides nice, non-stacked fractions and better spacing
% for units.
\usepackage{units}

% The fancyvrb package lets us customize the formatting of verbatim
% environments.  We use a slightly smaller font.
\usepackage{fancyvrb}
\fvset{fontsize=\normalsize}

% Small sections of multiple columns
\usepackage{multicol}

\begin{document}

\maketitle

%%\section{Description}

\section{Description}
\noindent The era of Facebook, MySpace and Twitter finds us connected
us in ways unimaginable even 10 years ago.  Google makes virtually
everything online eternally available while allowing us to walk down
virtual streets half a world away.  Your  smart phone knows where you
are and often what you are doing.  What expectations of privacy and
security are realistic in this age of hyper-connectivity?  Who owns
information about you?  How can technology help establish and deliver
trustworthy systems to society?  Our tutorial will explore ethical,
philosophical, and technical aspects of identity, privacy and trust in
digital systems.  We will examine how concepts surrounding trust are
defined and relate those concepts to technical and social mechanisms
for achieving them.  Concurrently, we will explore the societal
ramifications resulting when our systems violate the trust we place in
them.

%\section{Course Information}

\begin{margintable}
\begin{tabular}{ll}
  \textbf{Class Time}: & 4:00-4:50 T \\
  \textbf{Location}: & 102 Nunemaker \\
  \textbf{Instructor}: & Dr. Perry Alexander \\
  & 2022 Eaton Hall - 864-8833 \\
  & 136 Nichols Hall - 864-7741 \\
  & \url{palexand@ku.edu} \\
  & \url{http://www.ittc.ku.edu} \\
  \textbf{Office Hours}: & 10:00-11:00 TR, 2022 Eaton Hall \\
  & or by appointment \\
\end{tabular}
\end{margintable}

\section{Prerequisites}

Acceptance into the KU Honors Program

\section{Texts}

There are no formal texts for this class.  We will use papers from the
literature and some informal handouts that I provide.

\section{Grading}

Semester grades will be assigned on a standard 10 point scale.  I may
curve final grades at the end of the semester.  However, I will never
curve individual assignments or exams.  If I curve and how much I
curve is at my discretion.  However, I will never curve up -- 90\% and
above will always be an ``A''.
\begin{margintable}
  \begin{tabular}{ll}
    90-100\% & A \\
    80-90\% & B \\
    70-80\% & C \\
    0-70\% & F \\
  \end{tabular}
\end{margintable}

The sole basis for your grade in this course is: (i) preparation and
participation in our weekly meetings; and (ii) a moderate sized
project.  I do not take attendance, but it's pretty difficult to hide.
You will be asked to lead discussion one or more times during the
semester and there will be several very small projects.

\section{Homework}

Some preparation for class will be required each week.  You will be
asked to do some reading, some research, and learn some survival
skills for success at KU.  You may be asked to take charge of the
discussion from time to time, but never alone.

\section{Projects}

You will be asked to do a moderate sized project towards the end of
the semester.  The structure and topic of this project will largely
depend on what you want to study further.  More details will be
provided as the course progresses.

\section{Web Repository}

All homework assignments, exams, solutions and handouts you receive in
class are linked to my HNRS 190 homepage.
\marginnote{\footnotesize
  \url{http://www.ittc.ku.edu/~alex/teaching/hnrs190/}
  }
In general, I will not distribute hard copies of labs and homework
assignments in class.  All documents will be published using the Adobe
PDF standard.  PDF readers are freely available for Windows, Linux,
and MacOS on the Adobe website.

\section{Policies}

Generally, I am quite easy to get along with and I will help however I
can.  However, I do have are a few things that you should keep in mind
during the class that will help us get along.\footnote{The 4
  Commandments:
    \begin{itemize}
      \parskip=0pt\itemsep=0pt
    \item Do not whine
    \item Do not cheat
    \item Do not disrespect the TAs
    \item Do not tell me what you don't need to know
    \end{itemize}}

\newthought{Class Participation} I do not take attendance in class, however
participation in class is important to its success. Please ask
questions and participate in class discussions. When assigning final
grades, borderline cases will be decided based on class
participation. 

\newthought{Email} I encourage you to use email to contact me -- it is
by far the easiest way to find me. I am logged in when I am working
and check my mail frequently.

\newthought{Blog} The course blog is available on the website and via
an RSS feed.  I will post late-breaking news about projects, homework
and class administration on the blog.  Either subscribe, or check the
website frequently.

\newthought{Phone} Feel free to call me in my office at any time.  I
would prefer not to be called at home.

\newthought{Office Hours} I will make every effort to be in my office
during scheduled office hours. If there are exceptions, I will let you
know as early as is possible. If you have a conflict with my office
hours, please make an appointment or stop by my office at another
time. I have an open door policy, you are free to come by whenever you
choose. If I am busy, I may ask that you come back later, but please
don't hesitate to knock! My schedule is available online.

\newthought{Cheating} Academic misconduct of any kind will
automatically result in a 0 score on the homework, lab, project, or
exam in question and your actions will be reported to the department
chair.  Your homework, exams and projects must be individually
prepared unless otherwise noted.  Posting your assignments to Internet
discussion lists is considered academic misconduct. Sharing your
solutions with others is considered academic misconduct.  Turning in
solutions from previous semesters is considered academic misconduct.
Paying people to prepare solutions is academic misconduct.  Automated
mechanisms are available for checking the originality of source
code. Please spend your time trying to solve assigned problems rather
than trying to get around the system.  Don't risk it!

\newthought{Excuses} Excusing a missed exam or assignment is left to my
discretion. Illness, family emergencies, and religious observances are
examples of acceptable excuses. Computer down time, over sleeping,
political activities, and social events are examples of unacceptable
excuses. Please try to let me know of problems in advance when
possible and be prepared to provide verification of your excuse.

\newthought{Extensions} As a policy, I do not extend due dates of
homework and projects.  If I choose to do so, I will only announce the
extension in class, via email or on the blog.  If you hear an
extension has been granted and I have not announced it, your
information is incorrect.  Remember that if I grant extensions early
in the semester, it will necessarily compress due dates the end of the
semester.

\end{document}

